%                                                                 aa.dem
% AA vers. 8.2, LaTeX class for Astronomy & Astrophysics
% demonstration file
%                                                       (c) EDP Sciences
%-----------------------------------------------------------------------
%
%\documentclass[referee]{aa} % for a referee version
%\documentclass[onecolumn]{aa} % for a paper on 1 column  
%\documentclass[longauth]{aa} % for the long lists of affiliations 
%\documentclass[rnote]{aa} % for the research notes
%\documentclass[letter]{aa} % for the letters 
%\documentclass[bibyear]{aa} % if the references are not structured 
% according to the author-year natbib style

%
\documentclass[twocolumn,times]{aastex62}  %should match prefix of .cls file
%
\usepackage{graphicx}
\usepackage{float}
\usepackage{color}
%\usepackage{natbib}
%\usepackage{ulem}
%\usepackage{subfigure}
%\usepackage{subcaption}
%\usepackage[utf8]{inputenc}
%\usepackage[english]{babel}
\usepackage[colorinlistoftodos]{todonotes}

\newcommand{\smc}[1]{{\color{blue}$\blacksquare$~\textsf{[SMC: #1]}}}

%%%%%%%%%%%%%%%%%%%%%%%%%%%%%%%%%%%%%%%%
%\usepackage{txfonts}
%%%%%%%%%%%%%%%%%%%%%%%%%%%%%%%%%%%%%%%%
%\usepackage[options]{hyperref}
% To add links in your PDF file, use the package "hyperref"
% with options according to your LaTeX or PDFLaTeX drivers.
%
\begin{document} 


\title{Features of Late Time Gravitational Wave Emission from Two-dimensional Rotating Core-Collapse Supernovae}

\author{Michael A. Pajkos}
\affiliation{Department of Physics and Astronomy, Michigan State University, East Lansing, MI 48824, USA}
\affiliation{Department of Computational Mathematics, Science, and Engineering, Michigan State University, East Lansing, MI 48824, USA}

\author[0000-0002-5080-5996]{Sean M.~Couch}
\affiliation{Department of Physics and Astronomy, Michigan State University, East Lansing, MI 48824, USA}
\affiliation{Department of Computational Mathematics, Science, and Engineering, Michigan State University, East Lansing, MI 48824, USA}
\affiliation{National Superconducting Cyclotron Laboratory, Michigan State University, East Lansing, MI 48824, USA}
\affiliation{Joint Institute for Nuclear Astrophysics-Center for the Evolution of the Elements, Michigan State University, East Lansing, MI 48824, USA}

\author{Kuo-Chuan Pan}
\affiliation{Department of Physics and Institute of Astronomy, National Tsing Hua University, Hsinchu 30013, Taiwan}


\received{\today}
%\revised{September 27, 2016}
%\accepted{\today}
%% Command to document which AAS Journal the manuscript was submitted to.
%% Adds "Submitted to " the arguement.
\submitjournal{ApJ}
\shorttitle{Gravitational Waves from CCSNe}
\shortauthors{Pajkos et al.}

% \abstract{}{}{}{}{} 
% 5 {} token are mandatory
 \begin{abstract}
  We explore the influence of progenitor mass and rotation on the gravitational wave (GW) emission from core collapse supernovae.  We present the results from 15 two-dimensional (2D) neutrino radiation-hydrodynamic simulations from initial stellar collapse to $\sim$300 ms after core bounce.  We examine the features of the GW signals for four zero age main sequence (ZAMS) progenitor masses ranging from 12\(M_\odot\) to 60\(M_\odot\) and four core rotation rates from 0 rad s$^{-1}$ to 3 rad s$^{-1}$. We find that GW strain immediately around core bounce is fairly independent of ZAMS mass and---consistent with previous findings---that it is more heavily dependent on the core angular momentum.  At later times, all nonrotating progenitors exhibit loud GW emission, which we attribute to vibrational g-modes of the neutron star excited by convection in the post-shock layer.  We find that increasing rotation rates results in {\it muting} of the late-time GW signal due to centrifugal effects that inhibit convection in the post shock region, quench the SASI, and flatten the neutron star vibrational modes.  Additionally, we verify the efficacy of our approximate general relativistic (GR) effective potential treatment of gravity by comparing our core bounce GW strains with the 2D GR results of \citet{richers:2017}. 
   \keywords{gravitational waves --
                core-collapse --
                supernova
               }
\end{abstract}




%
%________________________________________________________________

\section{Introduction}

Core-collapse supernovae (CCSNe) became the first extra-solar multi-messenger object when SN 1987A was detected by the Kamiokande experiment in 1987 \citep{hirata:1987} along with concurrent electromagnetic observations \citep[cf.][]{arnett:1989}. With the recent detection of a neutron star merger--GW170817--in both photons and gravitational waves (GWs) by the LIGO and VIRGO collaborations \citep{abbott:2016} we have entered the era of {\it GW} multimessenger astronomy.  
So far, only the mergers of black holes binaries and a neutron star binary have been detected in GWs, but CCSNe are also predicted to be prodigious GW sources, though not quite as ``loud'' compact object binary mergers.
Accurate prediction of the expected GW signal from CCSNe is key to increasing the likelihood of detection by GW observatories such as aLIGO and VIRGO and will be crucial in our ability to extract physical meaning from a future CCSN GW detection.

CCSNe are routinely observed in the EM window, and the looming prospect of future synoptic surveys such as LSST and ZTF may increase the volume of such data for CCSNe by orders of magnitude.
Still, until the late nebular phase, which is often too dim to easily observer for distant CCSNe, the EM emission arises from the very outermost layers of the progenitor star and the central core regions, where the explosion is driven, are obscurred. 
This makes it challenging to connect EM emission from CCSNe directly to the mechanism that powers them.
Due to their relatively small interaction probabilities with matter, both neutrinos and GWs offer windows through which to peer directly into the heart of a CCSN explosion.  
Moreover, these observations yield numerous applications both inside and outside of astrophysics: restricting nuclear equations of state, probing theoretical quark-gluon matter within neutron stars, and providing strong field tests for the theory of general relativity--to name a few. \smc{I do not understand what you are trying to say in this sentence.}

When used in conjunction with EM signals, either gravitational waves or neutrino bursts offer
astrophysicists the opportunity to further constrain the physics of these aforementioned applications.  Yet, never before have all three sources been utilized, simultaneously.  Albeit a rare event, a Galactic, core-collapse supernova (CCSN) offers the perfect opportunity to observe the first ever `trifecta' of astrophysical transients.  With such a wealth of information available within core-collapse gravitational wave signals, accurate time-domain wave forms are vital to the LIGO and VIRGO collaborations.
\par
With high performance computing resources (HPC) becoming more accessible, the accuracy to which gravitational wave forms can be calculated is constantly being refined.  Nevertheless, due to the high degree of non-linearity in both Einstein's field equations and the equations of radiation-magnetohydrodynamics, approximations that preserve numerical accuracy become necessary, in order to minimize computational cost.  The conformal flatness condition (CFC) is an effective treatment of gravity that has been implemented.  Within a few percent, CFC accurately produces prebounce and early post bounce signals from CCSN \citep{ott:2007}.  The relativistic effective potential satisfies the solution to hydrostatic equilibrium according to the Tolman-Oppenheimer-Volkoff equation (TOV) \citep{marek:2006}.  Moreover, \citet{rampp:2002} condensed the expression into a compact relativistic potential that is less computationally intensive than CFC.  
\par
Historically, research into gravitational wave emission from rotating CCSNe has focused on the bounce and early post-bounce phase of the explosion; these investigations have found that increasing angular momentum of the core leads to a larger strain peak at bounce \citep{muller:1982,moench:1991,yamada:1995,zwerger:1997,dimm:2002,kotake:2003,shibata:2004}.  More recent investigations examine the role of the angular momentum distribution within the supernova progenitor and find it only important in the rapid rotation regime, where the ratio of kinetic to gravitational potential energy ($T/|W|$) $ \gtrsim 8\%$ at bounce \citep{abdik:2014}. In order to examine GW emission at later times, different groups have considered other factors for nonrotating cases---for example, convection in the post shock region \citep{burrows:1996,muller:1997,muller:2004,marek:2009b} and the standing accretion schock instability (SASI) \citep{blondin:2003,blondin:2006,ohnishi:2006,foglizzo:2007,scheck:2008,iwakami:2009,fernandez:2010}.  \citet{moro:2018} investigate late time gravitational wave emission  for only moderate rotational speeds ($\Omega_{core} = 0.2$ rad s$^{-1}$) and a single progenitor mass ($13 M_\odot$).  Also, \citet{pan:2018} investigate the relationship between black hole formation and gravitational wave emission, for a nonrotating 40 $M_\odot$ progenitor.

\par
In our paper, we present 15 axisymmetric (2D) hydrodynamic simulations.  Our parameter space includes four progenitor masses ranging from $12M_\odot-60M_\odot$ and four integral rotation velocities: $0-3 \text{ rad s}^{-1}$ \citep{Suk:2016}.  We note that while \citet{woosley:2006} observe that only 1\% of massive stars may reach the rapid rotation regime, that in certain, low metallicity cases, angular momentum loss due to stellar winds and magnetic braking is minimal \citep{yoon:2005}.  This fact demonstrates that the likelihood of rapidly rotating, type II, supernova progenitors may be slightly more common than previously thought.  However, when placed in the context of gravitational wave observations, we expect magnetic breaking due to the Taylor-Spruit Dynamo to slow the progenitor rotation \citep{spruit:2002}.  

%place in context of rotation
%TS dynamo breaks core -->0-1 regime
%GW observers should look here
%Drop in low frequencies between 0->1 rad/s

\par What is unique about our study is that, on top of the breadth parameter space covered, we consider the role of rotation into the late post bounce regime (300 ms pb).  We find that rotation restricts the dipole nature of SASI by centrifugally flattening the progenitor poles, thereby leaving it slightly oblate. Likewise, the positive angular momentum gradient created by the rotation stabilizes the post shock convection via the Solberg-H{\o}iland instability criterion \citep{endal:1978,fryer:2000}.  Not only are the SASI and post shock convection contributions to the gravitational radiation diminished, but the PNS vibrational signals are damped because of reduced downflow of matter onto the PNS surface.  \par 

\begin{table}[t]
\begin{tabular}{c|c|c}
Progenitor Mass [$M_\odot$] & Compactness & \textit{A}[cm] \\
\hline

12  & 0.0738 &         8.123e+07             \\
20  & 0.2785 &         1.021e+08            \\
40  & 0.5341 &         1.282e+08           \\
60  & 0.1708 &         9.112e+07          
\end{tabular}
\caption{Listed values for ZAMS mass, compactness calculated from \citet{Suk:2016}, and differential rotation parameter \textit{A}.}
\label{table:compact}
\end{table}



As our gravitational treatment approximates the mass distribution as an $l=16$ multipole, under the relativistic effective potential, our work also serves as a code comparison to general relativistic treatments of gravity \citep{richers:2017}.  We report our simulations produce nearly identical GW bounce signals to those that impose general relativistic gravity.  \par 
This study also serves as a cautionary tale when considering initial angular momentum distributions.  When applying rotational velocities to different radial, density profiles, they are naturally endowed with different amounts of angular momentum.  This difference can lead to modified gravitational wave bounce signals.    To avoid this discrepancy certain groups have adopted assigning specific angular momentum distributions \citep[eg.][]{oconnor:2011}.  Nevertheless, we adopt this convention for the sake of code comparison to other works that have employed similar angular velocity prescriptions \citep{richers:2017}.
\par
Our paper is laid out as follows:  In Section \ref{sec:method} we review our methods and treatment of microphysics within our \texttt{FLASH} simulations.  In Section \ref{sec:results} we address which progenitors explode, verify our gravitational treatment, analyze our simulated gravitational wave signals, and discuss detectability.  In Section \ref{sec:summary} we conclude and discuss.

%__________________________________________________________________

\begin{figure}[t]
    \centering
    \includegraphics[scale=0.45]{figures/omega_vs_r_m12.png}
    \caption{The rotation profiles for five of the \citet{heger:2005} progenitors.  Each solid line represents \citeauthor{heger:2005}'s (\citeyear{heger:2005}) model and the dashed lines are Equation (\ref{eq:omega}) applied to the respective progenitor, with the appropriate differential rotation parameter.  }
    \label{fig:ovsr}
\end{figure}

\section{Methods and Simulation Setup}
\label{sec:method}
We run 2-D simulations with \texttt{FLASH} version 4 \footnote{http://flash.uchicago.edu/site/}, a multiscale, multiphysics code, made available through the FLASH Center for computational science. \citep{fryxell:2000,dubey:2009}.  We employ a modified relativistic, effective potential, treating the mass distribution as an $l=16$ multipole, incorporating the multipole Poisson solver of \citet{couch:2013a}, where $Y_l^m$ refers to the spherical harmonics.  We utilize the SFHo equation of state (EOS) throughout our 15 simulations \citep{steiner:2013}.  Our grid setup is a 2D cylindrical construction with the PARAMESH (v.4-dev) library for adaptive-mesh-refinement (AMR) \citep{macneice:2000}.  The outer radial boundary is $1.0 \times 10^4$ km, with nine levels of refinement.  \par
%M1 section
Neutrinos play a vital role in core-collapse supernovae.  Directly after collapse, they provide an avenue through which the protoneutron star can cool.  As the shock propagates outwards, they also provide heating in the gain region that is crucial in reviving the explosion, according to the neutrino heating mechanism.  The opacity of the material to these outflowing neutrinos also must be carefully accounted for in an energy dependent way.  We incorporate a multidimensional, multispecies, energy dependent, two moment scheme with an analytic closure, or the so called M1 scheme.  Our implementation is based on \citet{oconnor:2015}, \citet{shibata:2011}, and \citet{cardall:2013}.  A detailed outline of the M1 implementation in \texttt{FLASH} is in \citet{oconnor:2018}.  In order to reduce computational cost to explore the wide parameter space for our study, we neglect velocity dependent neutrino transport and do not account for inelastic, neutrino-electron scattering.

\subsection{Treatment of Rotation}



Our models initially rotate according the rotation relation, 
\begin{equation}
    \Omega'(\omega) = \Omega \bigg[1 + \bigg(\frac{\omega}{A}\bigg)^2 \bigg]^{-1}, 
    \label{eq:omega}
\end{equation}
where $\omega$ is the distance from the rotation axis, $\Omega$ is the initial rotational velocity of the star, and $A$ is the differential rotation parameter \citep{eriguchi:1984}.  For large values of $A$, the stellar rotation is nearly solid body, whereas small values of $A$ lead to a differential profile.  \par Historically, $A$ values have been explored in parameter space, or a single value is chosen for different progenitor masses.  Motivated by stellar compactness, we choose unique $A$ values for each progenitor mass.  In order to choose an appropriate $A$ value, we begin with radial, rotation profiles from \citet{heger:2005} of 12, 15, 20, 25, and 35 $M_{\odot}$ stars, with various angular momentum transport parameters.  Using the \texttt{curve\_fit} function (in the \texttt{scipy.optimize} library) available in \texttt{Python}, we obtain $A$ values that correspond to the most accurate fits of Equation (\ref{eq:omega}) to 20 of \citeauthor{heger:2005}'s \citeyear{heger:2005} rotation profiles.  Figure \ref{fig:ovsr} displays the radial, rotation profile for five of the aforementioned progenitors, compared to our implementation of Equation (\ref{eq:omega}), with the appropriate $A$ value.


\begin{figure}[t]
    \centering
    \includegraphics[scale=0.45]{figures/a_vs_compact.png}
    \caption{Linear relation between differential rotation parameter, $A$, and compactness parameter of the inner 2.5 $M_\odot$, $\xi_{2.5M_\odot}$.  The linear trend is constructed from the \citet{heger:2005} rotation profiles.  We then apply the relation to the compactness values from \citet{Suk:2016} to yield the differential rotation parameters.  The progenitor ZAMS masses are labeled in units of $M_\odot$ for each respective point.}
    \label{fig:a_vs_comp}
\end{figure}

\par The 20 differential rotation parameters are then compared to the compactness, $\xi$, of each progenitor.  We define compactness similar to \citet{oconnor:2011}, 
\begin{equation}
    \xi = \left.\frac{M/M_{\odot}}{R(M_{bary}=M)/1000 \text{km}}\right\vert_{collapse} ,
\end{equation} 
where we choose $M = 2.5 M_\odot$, and $R(M_{bary}=M) $ as the radius at which the internal baryonic mass is $2.5M_\odot$, at collapse.  A linear relation is then obtained of $A$ vs. $\xi$ (blue stars in Figure \ref{fig:a_vs_comp}).  By then applying the calculated compactness values from the four \citet{Suk:2016} progenitors, we determine the  optimal $A$ values (orange circles in Figure \ref{fig:a_vs_comp}).  Likewise, the respective progenitor ZAMS mass is labeled in units of $M_\odot$ for each corresponding point in Figure \ref{fig:a_vs_comp}.  For a full list of the progenitor masses, compactness values, and A's, see Table ~\ref{table:compact}.  As a note, we choose to omit the 40 $M_\odot$ progenitor at $\Omega = 3$ rad s$^{-1}$ from our analysis with a physically motivated rationale.  The $\xi$ value of this progenitor is nearly double that of the 20 $M_\odot$ progenitor (the next closest compactness value).  This fact displays that the 40 $M_\odot$ has vastly more mass within a 2.5 $M_\odot$ mass radius compared to the other progenitors and will therefore be endowed with more angular momentum.  We outline in greater detail the resulting distortion of the GW signal in Section \ref{sec:results}.  Thus, in order to preserve a maximum upper limit on the angular momentum (within a 1.75 $M_\odot$ mass radius) of $\sim 2.4\times 10^{49}$ erg s, we omit the 40 $M_\odot$ from our analysis.  \\

% 1) 12,15,20,25,35 fit XXX profiles to toy model
% 2) plot optimal A values to compactness from XXX
% 3) plot linear relationship
% 3) Extrapolate to higher masses (w/ given compactness) from XXX


%
%______________________________________________________________
\begin{figure}[t]
    \centering
    \includegraphics[scale=0.40]{figures/M1_shock_mass.png}
    \caption{Shock radius evolution of the four  progenitor models versus time (post bounce).  Similar to \citet{couch:2013b}, we define a successful explosion as the average shock front exceeding 400 km. }
    \label{fig:shock}
\end{figure}

\section{Results}
\label{sec:results}

To extract the gravitational wave signal from our simulations, we adopt the dominant, quadrupole moment formula for the gravitational strain, through the slow motion, weak field formalism %\citep[eg.][]{misner:1973,murphy:2009}, 
\citep[eg.][]{finn:1990,blanchet:1990}
\begin{equation}
    h_+ \approx \frac{2G}{Dc^4}
    \frac{d^2I_{zz}}{dt^2},
\label{eq:quad}
\end{equation}
where $I_{zz}$ is the mass quadrupole moment, $G$ is the gravitational constant, $c$ is the speed of light, $D$ is the distance to the source (our fiducial value is $D=10$ kpc), and we assume optimal detector orientation.\\
\par When plotting the amplitude spectral density (ASD) of the gravitational wave signal we compute the discrete fourier transform (DFT) consistent with \citet{anderson:2004} and LIGO's implementation,
\begin{equation}
\widetilde{h}_{+k} = \sum^{N-1}_{j=0} h_{+j} e^{-i2\pi jk/N}
\label{eq:dft}
\end{equation}
where $i=\sqrt{-1}$.

\subsection{Rotation's Influence on Explosions}
\par With respect to successful supernova explosions, we report only two nonrotating progenitors (20 $M_\odot$ \& 60 $M_\odot$) explode, where we define a successful explosion as the average shock exceeding 400 km and not receding below that value.  From photodissociation to entropy increases due to convection, the propagation of the outward shock front involves a variety of microphysics; thus, the effect of rotation on a successful explosion is not a simple one.  In one respect, one expects greater centrifugal support to lead to a larger shock front.  However, there are two factors that inhibit the shock from propagating outward.  The first is the inhibited convection due to the positive angular momentum gradient within the progenitor.  By inhibiting convection in the gain region, the shock front loses vital heating that would help power a successful explosion.  The  second rotational element that inhibits explosions is the lack of neutrino production.  A centrifugally supported, rotating progenitor will have a slower matter infall during collapse.  As the infalling material has less kinetic energy, it will create a less dense PNS.  Furthermore, a less energetic infall will in turn create a weaker bounce and shock front.  The combination of a less dense PNS, paired with less photodissociation, due to a weaker shock, creates an unfavorable scenario for a supernova explosion that is revived by neutrino heating.  Hence, the introduction of rotation involves competing forces that can enhance or diminish the shock.  Figure \ref{fig:shock} illustrates the average shock radius evolution versus time (post bounce). Clearly this evolution is not monotonic with increasing rotational velocity.

 \begin{figure}[t]
    \centering
    \includegraphics[width=0.5\textwidth]{figures/bounce_richers_final.png}
    \caption{A plot of gravitational strain vs time (post-bounce) for a 12\(M_\odot\) progenitor with $\Omega = 3$ rad s$^{-1}$.  Plotted in the dashed line is the GW strain from the CFC \texttt{CoCoNuT} code and the solid line is modeled by the relativistic effective potential.  While the different treatments of hydrodynamics lead to differences in the strain in the early post bounce phase, we qualitatively verify our gravitational treatment by obtaining a nearly exact bounce signal. }
    \label{fig:bounce_cfc}
\end{figure}

\subsection{Gravitational Comparison}


\begin{figure*}[t]
  \centering     %%% not \center
  \includegraphics[width=0.48\textwidth]{figures/hd3_bounce_final.png}
  \includegraphics[width=0.48\textwidth]{figures/hdj_bounce_final.png}
  \caption{(Left) Gravitational wave bounce signal from all 10 progenitor masses with $\Omega_o = 3 \text{ rad s}^{-1}$.  By applying Equation (\ref{eq:omega}), we assign a radially dependent, angular velocity to our progenitors.  Because the central density profiles of each progenitor are different---namely a lower density $12 M_\odot$ and higher density $40 M_\odot$---the progenitor cores are endowed with different amounts of angular momenta.   (Right) Modified bounce signals after adjusting rotation rates to yield similar angular momenta ($\sim 2.4\times10^{49} \text{erg s}$) of the inner $1.75 M_\odot$ of matter.  As predicted by \citet{dimm:2008} and \citet{abdik:2010,abdik:2014}, the GW bounce signals depend on the inner core angular momentum at bounce, not the original ZAMS mass.}
  \label{fig:bounce}
\end{figure*}

When considering multi-D simulations, the treatment of gravity must offer a balance between numerical accuracy, while considering computational cost.   The conformal flatness condition offers a nearly identical gravitational wave signal, compared to full general relativity, while reducing simulation time \citep{ott:2007}.  Figure \ref{fig:bounce_cfc} offers a qualitative check of the relativistic effective potential, obeying the TOV equation, versus conformal flatness \citep{marek:2006,richers:2017}.  We incorporate an identical deleptonization profile and SFHo equation of state \citep{steiner:2013}.  For this comparison, we match the neutrino physics of their simulation by using a ray-by-ray, three species, neutrino leakage scheme \cite{oconnor:2010,couch:2014}.  We capture a nearly identical bounce signal and similar strain up to 5 ms post bounce. \par
However, after the initial bounce signal ring-down, it is clear to see the different computational treatments of hydrodynamics produce a difference in the post bounce convection and the shock outflow, thereby modifying the GW strain.  Although not exact, the efficiency of the relativistic effective potential offers a computationally favorable method to accurately model the gravitational wave signal for a myriad of progenitors, and acts as a qualitative check to the GW strain in the late post-bounce regime.\\

%-----------------------------------
%-----------------------



\subsection{ZAMS Influence on Gravitational Bounce Signal}

\begin{figure*}[t]
\includegraphics[width=\textwidth]{figures/ccsn2D_M1_all.png}
\centering
\caption{Time domain wave forms over our entire parameter space.  For all four progenitor masses, the centrifugal muting of the gravitational wave signal is clear in the late post bounce regime.    Furthermore, we do not see any relation between ZAMS mass and late time GW emission.}
\label{fig:ccsn_all}
\end{figure*}

While unique progenitors $\gtrsim 8 M_\odot$ will experience different amounts of mass loss that can depend on ZAMS mass,  their iron core's will all reach the Chandrasekhar limit.  This nearly identical core mass across ZAMS parameter space yields similar core angular momenta.  Hence, the core bounce signal is nearly indistinguishable between progenitor masses.  For further verification of our gravitational treatment, we perform 20 additional simulations---from collapse---until eight milliseconds after core bounce, in order to replicate this bounce signal degeneracy.  Outlined by \citet{ott:2012}, neutrino leakage has a small effect on the GW bounce and early post bounce signal.  Hence, to expedite our simulations we employ the same leakage scheme mentioned above.  Moreover, our results are consistent with 3D, fully general relativistic predictions given by \citet{ott:2012}, that similar core angular momenta yields similar GW bounce signals.  

Figure \ref{fig:bounce} displays the bounce signals for all 10 progenitor masses.  The left panel is for uniform rotational velocity prescriptions at $\Omega = 3\text{ rad s}^{-1}$.  As previously highlighted, the angular momentum of the inner core is the main contributor to the gravitational bounce signal.  While many of the wave forms have similar amplitudes, there are two clear outliers: the $12 M_\odot$ and $40 M_\odot$ progenitors.  The $12 M_\odot$ and $40 M_\odot$ progenitors respectively have lower and higher central densities at collapse, by nearly a factor of two.  Because we endow each progenitor with angular velocity, and not specific angular momentum, the denser $40 M_\odot$ progenitor will receive more angular momentum, compared to the remaining progenitors, thereby affecting the morphology of the iron core and the resulting GW bounce signal.  As outlined by \citet{dimm:2008}, once a star is sufficiently rotating, the centrifugal support slows the bounce, diminishing the GW bounce amplitude and widening out the bounce peak of the waveform.  

The inverse is true for the $12 M_\odot$ case.  Because it has a less dense, inner core at collapse, it will receive less initial angular momentum, thereby producing a lower amplitude bounce signal.  After modifying the initial rotation rates to match the progenitor core angular momenta (right panel of Figure \ref{fig:bounce}) the change produces nearly identical GW bounce signals.  

Hence, our results from exploring the bounce signal over a wide range of progenitor masses validate the previous discoveries of angular momentum dependence of the gravitational wave signal of \citet{dimm:2008} and \citet{abdik:2010,abdik:2014}, but also serve as a cautionary note for future groups who perform rotating, core collapse simulations, who use a wide variety of progenitor models.  It is worth noting that other rotational treatments exist beyond the simple angular velocity law, such as specifying a radial, specific angular momentum profile \citep[eg.][]{oconnor:2011}.  

We validate our use of the relativistic effective potential by capturing the bounce signal, compared to CFC, and by replicating the degeneracy of bounce signals, observed by 3D, fully general relativistic simulations.  Having passed these two tests, we advance into the post bounce regime to examine rotational effects on the GW signal out to 300 ms after bounce.


\begin{figure*}[htp]
  \centering     %%% not \center
  \includegraphics[width=\textwidth]{figures/tdwf_region_20.png}
  \caption{Time domain waveforms for the 20 $M_\odot$ progenitor.  Each panel corresponds to the region from which the gravitational waves are emitted.  The large contribution in the top panel indicates the main source of late time gravitational waves is from the vibrating neutron star.  The middle panel displays the inhibited convective signal $\sim 50-100 $ ms post bounce that is characteristic of this quiescent phase.}
  \label{fig:region}
\end{figure*}


\begin{figure}[t]
    \centering
    \includegraphics[scale=0.38]{figures/gws_2x2_line_M1.png}
    \caption{Spectrograms for the $12 M_\odot$ progenitor over all four rotational velocities.  The key aspects revealed by the spectrogram are the rotational muting of gravitational waves and the flattening of the signal from the surface g-mode of the PNS.  This flattening is a product of the enlarged radius of the neutron star due to centrifugal effects and can be characterized by the dynamical frequency ($f_{dyn} = \sqrt{G \overline{\rho}}$), overlaid in grey.}
    \label{fig:2x2}
\end{figure}


\subsection{Rotational Influence on Late Gravitational Wave Emission}


While the consistency of the Chandrasekhar mass for collapsing iron cores creates a setting where envelope mass has little effect on the bounce signal, the post bounce dynamics of the explosion largely depend on the mass surrounding the newly born neutron star.  For nonrotating CCSNe, the shock front propagates outward, and loses energy through conversion of kinetic to gravitational potential energy and through hydrodynamical damping by the progenitor envelope.  However, in the case of rotation, the initial progenitor and resulting shock front become more oblate.  This more compact envelope will affect the gravitational wave emission in three respects: the post shock convection is damped, the SASI becomes restricted, and the PNS vibrational modes are inhibited. 

 After the infalling matter from collapse reaches nuclear densities, the core nuclei dissolve into nucleons, allowing the strong force to become repulsive and reverse the material infall.  On the time scale of tens of microseconds, the subsonic inner core encounters the supersonic outer core, forming a shock front.  As this shock front photodissociates overlying material and releases an enormous neutrino flux, it leaves behind a negative entropy gradient \citep{mazurek:1982,bruenn:1985,bruenn:1989}.  This scenario is unstable according to the Ledoux Criterion, causing prompt convection in the post shock region \citep{burrows:1992} , therefore creating an associated emission of gravitational radiation \citep{marek:2009b,ott:2009}.  However, as rotation becomes more prominent in our models, a positive angular momentum gradient is established within the explosion, thus making it more convectively stable, by the Solberg-H{\o}iland instability criterion \citep{endal:1978,fryer:2000}.  This inhibited convection is in part responsible for reducing the GW amplitude at later times.  Figure \ref{fig:ccsn_all} clearly displays the trend of decreasing GW amplitude with increasing rotational velocity, over all progenitor masses.  Furthermore, we recast our analysis by focusing on regions within the explosion that emit GWs.  The top two panels of Figure \ref{fig:region} display the inhibited convective signal with increasing rotation, as the GW signal around $\sim$30 ms post bounce becomes increasingly muted.  The typical convective signals in the early post bounce regime are then quickly washed out by the post bounce ring down of the newly born neutron star, as rotation increases.   
\begin{figure*}[t!]
  \centering     %%% not \center
  \includegraphics[width=\textwidth]{figures/tbe6tbe300_M1_long.png}
  \caption{Amplitude spectral density (ASD) plot of all progenitors for all rotation rates from $t_{be}+6$ ms $\rightarrow t_{be}+300$ ms.  The rotational muting of the fundamental neutron star g-mode is displayed as the peak frequency ($\sim 800$ Hz) becomes less prevalent, with increasing rotation rate.  Likewise, the low frequency signals ($\sim30$ Hz) become more audible, with increasing rotational velocity.  Plotted in the black dashed line is the design sensitivity curve for aLIGO in the zero detuning, high sensitivity configuration \citep{barsotti:2018}.  The blue dotted line is the predicted KAGRA detuned, sensitivity curve \citep{komari:2017}.}
  \label{fig:spetra_long}
\end{figure*}

Under typical nonrotating conditions, the shocked accretion flow will grow unstable from the nonradial deformation modes and amplify to eventually become a dominant driver in reviving the shock; this effect is also known as SASI \citep{scheck:2008,marek:2009a}.  In 2D simulations, the SASI excites large, oscillatory flows along both poles that drives changes in entropy, capable of causing post shock convection.  It is worth noting in 3D simulations, the SASI can excite `spiral' modes that correspond to nonzero $m$ values \citep{kuroda:2016}.  The high degree of nonlinearity between the hydrodynamic flows, neutrino interactions, and gravitational effects can yield matter flow that is largely quadrupolar, thereby resulting in gravitational wave emission.  However, when the shock becomes restricted in the polar direction, due to centrifugal effects, the SASI cannot develop the diffuse bipolar outflow. The middle panel of Figure \ref{fig:region} illustrates the damping of the SASI contribution to GW emission.  For the nonrotating cases, the emission is largely stochastic in nature and becomes prominent around 150 ms after core bounce.  With increasing rotational velocity comes less pronounced, SASI, GW emission.  Note, while we expect the late time SASI activity to contribute uniquely to the GW spectrum, depending on progenitor mass, the rotational muting of the gravitational waves is universal across ZAMS mass parameter space.  Note \citet{burrows:2007}  and \citet{moro:2018} acknowledge the partial suppression of SASI, but the prior does not focus on the gravitational radiation emitted and the latter only examines a single, slow rotating, progenitor.  Our work marks the first study that supports the rotational muting of late time gravitational waves, over such a wide region of parameter space. 
 
 
With respect to protoneutron stars, a variety of oscillatory modes exist that could be of interest to current and future gravitational wave astronomers: fundamental f-modes, pressure based p-modes, and gravity g-modes---due to chemical composition and temperature gradients \citep{unno:1989}.  The typical frequency of the neutron star f-mode is around 1 kHz, and p-modes have frequencies greater than f-modes, giving gravitational wave astronomers little use, with current detector capabilities \citep{ho:2018}.  The frequencies of g-modes, however, are on the order of hundreds of Hertz, ideally falling within the detectability range of current gravitational wave detectors \citep{martynov:2016}.  The top panel of Figure \ref{fig:region} displays the contribution of the vibrating PNS to the majority of the late time GW signal, with $h_+D$ strain amplitudes around 50 cm.  These g-modes are thought to be excited by downflow from post shock convection or internal PNS convection \citep{murphy:2009,marek:2009b,muller:2013}.  Figure \ref{fig:2x2} is a spectrogram for the $12 \, M_\odot$ progenitor over all rotational speeds, where lighter colors represents greater strain amplitudes, $h_+$.  The dominant yellow band that extends from 100 Hz to 1000 Hz represents this contribution.  Overlaid in grey is the dynamical frequency, $f_{dyn} = \sqrt{G \overline{\rho}}$, that evolves synchronously with the g-mode contribution.  The synchronized evolution of $f_{dyn}$ and the frequency at which the PNS emits gravitational radiation is no coincidence.  As both are fundamentally related to the mass and radius of the neutron star, we expect that both are affected similarly when introducing rotation.  The initial progenitor rotation will centrifugally support the PNS thereby leaving it with a larger average radius.  Similar to two tuning forks of different length, the PNS with a larger radius will emit at a lower frequency, compared to a smaller PNS.  This `flattening' of the emitted frequency is displayed in Figure \ref{fig:2x2}.  Furthermore, Figure \ref{fig:2x2} provides a different lens through which the rotational muting is displayed, via the progressively darker panels with increasing rotational velocity.  We note that more robust peak GW frequency calculations exist \citep[eg.][]{muller:2013}, but include the $f_{dyn}$ relation to bring it to the attention of the reader for future estimates of the PNS frequency contribution to the GW signal.

We also Fourier transform the late time signal, as displayed in Figure \ref{fig:spetra_long} and scale the magnitude of the Fourier coefficients by $\sqrt{f}$ in order to produce amplitude spectral density plots.  These plots commonly display the sensitivity curves of current and next generation gravitational wave detectors.  We define $t_{be}$ similar to \citet{richers:2017} as the third zero crossing of the gravitational strain.  We choose to focus on the signal later than $t_{be} + 6$ ms in order to remove the bounce signal and early post bounce oscillation contribution to the signal.  The dominant contributions are the prompt convection, SASI, and surface g-modes of the PNS---as displayed by a peak frequency ranging from 700--1000 Hz.  Universally, the prevalence of the peak frequency decreases with increasing rotational velocity.  




\subsection{Observability of the Late Time Signal}

 Overlaid on our ASD plots is the expected performance of future gravitational wave observatories.  In the black dashed line and blue dotted line we have plotted the sensitivity curves of design sensitivity for Advanced LIGO in the zero detuning, high sensitivity configuration and the predicted KAGRA detuned sensitivity curve, respectively \citep{komari:2017,barsotti:2018}.
These curves represent the incoherent sum of the principal noise sources to the best understanding of the respective collaborations.  While these curves do not guarantee the performance of the detectors, they act as good guides for their anticipated sensitivities, nonetheless. 




Beyond the decreased prevalence of the peak frequency, an interesting trend emerges in Figure \ref{fig:spetra_long}, as rotation increases.  We highlight a noticeable difference in the amplitude of the low frequency contributions, particularly around 30 Hz.  The nonrotating progenitors have inaudible low frequency signals for the aLIGO and KAGRA detectors, whereas rotating progenitors create measureable signals at low frequencies.  \textit{This observation grants gravitational wave astronomers one of the first late time features that can help delineate progenitor angular momentum information.}  Whereas previous rotating, core collapse, gravitational wave studies have focused on the bounce signal as means to determine rotational features, or have focused on late time signals without rotation, our study unifies both facets, and opens the door to measuring late time gravitational wave signals that encode progenitor, angular momentum information. 

Because we plan to incorporate more detailed microphysics into our simulations, we refrain from making quantitative relations between low frequency spectra and progenitor angular momentum, in aims to conduct a more thorough investigation in future work.



\section{Summary and Conclusion}
\label{sec:summary}

We have explored the influence of rotation on the gravitational wave emission from CCSNe for four different progenitor masses and four different rotational velocities.  Having observed a linear relation between compactness, $\xi$, and differential rotation parameter, $A$, we calculate appropriate $A$ values for each progenitor mass, based on the compactness of the \citet{Suk:2016} progenitors.  Of our 15 simulations, only two nonrotating progenitors successfully explode, whereas the remaining rotating progenitors do not because of rotationally inhibited convection in the gain region and less neutrino production.  We notice the complex interplay between centrifugal support and neutrino heating as successful explosions do not display a monotonic relationship with rotation.

While there are more accurate treatments of gravity, we utilize the relativistic potential in order to streamline calculations, granting us the ability to explore larger sections of parameter space. Moreover, we validate our treatment by conducting two tests: the first matches the bounce signal of CFC gravity with general relativistic hydrodynamics \citep{richers:2017}.  The second agrees with 3D simulations incorporating fully general relativistic gravity, that display the degeneracy of bounce signals for cores with similar angular momenta \citep{ott:2012}.  

The main contributors to the late time gravitational wave signal (10--300 ms post bouce) are the post bounce convection, the SASI, and the surface g-modes of the neutron star.  By establishing a positive angular momentum gradient, the convection is suppressed according to the Solberg-H{\o}iland instability criterion \citep{endal:1978,fryer:2000}.  The more oblate shock front inhibits the bipolar outflow of the SASI.  Moreover, as the aforementioned mechanisms typically excite the g-modes of the PNS, vibrational emission from the PNS is also inhibited.  \textit{Thus, we confirm the muting of gravitational radiation due to rotation, in 2D simulations.}  

Before the PNS g-mode signal is completely muted, as rotation gradually increases, this signal gradually flattens and can be characterized by its dynamical frequency.  This observation is no coincidence as both  fundamentally depend on the neutron star's radius and mass.  With more centrifugal support, the newly formed neutron star forms with a larger radius.  This larger radius causes the surface of the neutron star to emit at lower frequencies, thereby producing a `flatter' signal.

We reveal a unique rotational effect on the late time GW signal.  \textit{We notice that the nonrotating progenitors all produce low frequency signals ($\sim 30$ Hz) that are inaudible, whereas the progenitors with larger angular velocities produce measureable GW signals for the aLIGO and KAGRA detectors.}  This observation gives gravitational wave astronomers one of the first features to investigate when trying to extract angular momentum information from late time gravitational waves.  We postpone asserting quantitative relations between low frequency emission and progenitor angular momentum until we incorporate more detailed microphysics into the \texttt{FLASH} framework.

While our approximations have allowed us to make large sweeps of parameter space, they leave room for us to include more robust microphysics.  In an ideal situation, we would compute 3D simulations, include full GR, magnetohydrodynamics, and GR Boltzmann neutrino transport that incorporates velocity dependence and inelastic electron scattering.  These additions would allow for more accurate gravitational wave forms and allow other phenomena to occur, for example the $m\ne 0$ modes of the SASI.


\acknowledgements

We would like to thank Evan O'Connor for his insightful perspective, as well as Jess McIver for pointing us to the aLIGO sensitivity curves.  This work was supported under the Michigan State University Distinguished Fellowship. 

    \software{\href{https://www.scipy.org/}{SciPy}}
    

%\bibliographystyle{aasjournal}%name of .bst file

%\bibliography{masterDB, ref} %name of .bib file
\bibliography{ref}

%-------------------------------------------------------------------

% \begin{thebibliography}{}



\end{document}
